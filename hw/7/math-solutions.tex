\documentclass[12pt,letterpaper]{article}
\usepackage{fullpage}
\usepackage[top=2cm, bottom=4.5cm, left=2.5cm, right=2.5cm]{geometry}
\usepackage{amsmath,amsthm,amsfonts,amssymb,amscd}
\usepackage{lastpage}
\usepackage{enumerate}
\usepackage{fancyhdr}
\usepackage{mathrsfs}
\usepackage{xcolor}
\usepackage{graphicx}
\usepackage{listings}
\usepackage{hyperref}
\usepackage{dsfont}

\hypersetup{%
  colorlinks=true,
  linkcolor=blue,
  linkbordercolor={0 0 1}
}
 
\renewcommand\lstlistingname{Algorithm}
\renewcommand\lstlistlistingname{Algorithms}
\def\lstlistingautorefname{Alg.}

\lstdefinestyle{Python}{
    language        = Python,
    frame           = lines, 
    basicstyle      = \footnotesize,
    keywordstyle    = \color{blue},
    stringstyle     = \color{green},
    commentstyle    = \color{red}\ttfamily
}

\setlength{\parindent}{0.0in}
\setlength{\parskip}{0.05in}

% Edit these as appropriate
\newcommand\course{DS-GA 1003}
\newcommand\hwnumber{5}                  % <-- homework number
\newcommand\NetIDa{Cody Fizette}           % <-- NetID of person #1
\newcommand\NetIDb{netid12038}           % <-- NetID of person #2 (Comment this line out for problem sets)
\newcommand*{\vertbar}{\rule[-1ex]{0.5pt}{2.5ex}}
\newcommand*{\horzbar}{\rule[.5ex]{2.5ex}{0.5pt}}

\pagestyle{fancyplain}
\headheight 35pt
\lhead{\NetIDa}
\lhead{\NetIDa}                 % <-- Comment this line out for problem sets (make sure you are person #1)
\chead{\textbf{\Large Homework \hwnumber}}
\rhead{\course}
\lfoot{}
\cfoot{}
\rfoot{\small\thepage}
\headsep 1.5em

\begin{document}
\global\long\def\reals{\mathbf{R}}
 \global\long\def\integers{\mathbf{Z}}
\global\long\def\naturals{\mathbf{N}}
 \global\long\def\rationals{\mathbf{Q}}
\global\long\def\ca{\mathcal{A}}
\global\long\def\cb{\mathcal{B}}
 \global\long\def\cc{\mathcal{C}}
 \global\long\def\cd{\mathcal{D}}
\global\long\def\ce{\mathcal{E}}
\global\long\def\cf{\mathcal{F}}
\global\long\def\cg{\mathcal{G}}
\global\long\def\ch{\mathcal{H}}
\global\long\def\ci{\mathcal{I}}
\global\long\def\cj{\mathcal{J}}
\global\long\def\ck{\mathcal{K}}
\global\long\def\cl{\mathcal{L}}
\global\long\def\cm{\mathcal{M}}
\global\long\def\cn{\mathcal{N}}
\global\long\def\co{\mathcal{O}}
\global\long\def\cp{\mathcal{P}}
\global\long\def\cq{\mathcal{Q}}
\global\long\def\calr{\mathcal{R}}
\global\long\def\cs{\mathcal{S}}
\global\long\def\ct{\mathcal{T}}
\global\long\def\cu{\mathcal{U}}
\global\long\def\cv{\mathcal{V}}
\global\long\def\cw{\mathcal{W}}
\global\long\def\cx{\mathcal{X}}
\global\long\def\cy{\mathcal{Y}}
\global\long\def\cz{\mathcal{Z}}
\global\long\def\ind#1{1(#1)}
\global\long\def\pr{\mathbb{P}}
\global\long\def\var{\textrm{Var}}
\global\long\def\cov{\textrm{Cov}}
\global\long\def\sgn{\textrm{sgn}}
\global\long\def\sign{\textrm{sign}}
\global\long\def\kl{\textrm{KL}}
\global\long\def\law{\mathcal{L}}
\global\long\def\eps{\varepsilon}
\global\long\def\convd{\stackrel{d}{\to}}
\global\long\def\eqd{\stackrel{d}{=}}
\global\long\def\del{\nabla}
\global\long\def\loss{\ell}
\global\long\def\tr{\operatorname{tr}}
\global\long\def\trace{\operatorname{trace}}
\global\long\def\diag{\text{diag}}
\global\long\def\rank{\text{rank}}
\global\long\def\linspan{\text{span}}
\global\long\def\proj{\text{Proj}}
\global\long\def\argmax{\operatornamewithlimits{arg\, max}}
\global\long\def\argmin{\operatornamewithlimits{arg\, min}}
\global\long\def\bfx{\mathbf{x}}
\global\long\def\bfy{\mathbf{y}}
\global\long\def\bfl{\mathbf{\lambda}}
\global\long\def\bfm{\mathbf{\mu}}
\global\long\def\calL{\mathcal{L}}
\global\long\def\vw{\boldsymbol{w}}
\global\long\def\vx{\boldsymbol{x}}
\global\long\def\vxi{\boldsymbol{\xi}}
\global\long\def\valpha{\boldsymbol{\alpha}}
\global\long\def\vbeta{\boldsymbol{\beta}}
\global\long\def\vsigma{\boldsymbol{\sigma}}
\global\long\def\vmu{\boldsymbol{\mu}}
\global\long\def\vtheta{\boldsymbol{\theta}}
\global\long\def\vd{\boldsymbol{d}}
\global\long\def\vs{\boldsymbol{s}}
\global\long\def\vt{\boldsymbol{t}}
\global\long\def\vh{\boldsymbol{h}}
\global\long\def\ve{\boldsymbol{e}}
\global\long\def\vf{\boldsymbol{f}}
\global\long\def\vg{\boldsymbol{g}}
\global\long\def\vz{\boldsymbol{z}}
\global\long\def\vk{\boldsymbol{k}}
\global\long\def\va{\boldsymbol{a}}
\global\long\def\vb{\boldsymbol{b}}
\global\long\def\vv{\boldsymbol{v}}
\global\long\def\vy{\boldsymbol{y}}
\global\long\def\minimizer#1{#1_{*}}
\global\long\def\ex{\mathbb{E}}
\global\long\def\nll{\text{NLL}}

\newenvironment{solution}[2][Solution]{\begin{trivlist}
\item[\hskip \labelsep {\bfseries #1}]}{\end{trivlist}}
\newenvironment{problem}[2][Problem]{\begin{trivlist}
\item[\hskip \labelsep {\bfseries #1}\hskip \labelsep {\bfseries #2.}]}{\end{trivlist}}


\section*{4. Multilayer Perceptron}

\begin{problem}{4.1.1.1}
    Show that $\frac{\partial J}{\partial W_{ij}}=\frac{\partial J}{\partial y_{i}}x_{j}$,
    where $x=\left(x_{1},\ldots,x_{d}\right)^{T}$. {[}Hint: Although
    not necessary, you might find it helpful to use the notation $\delta_{ij}=\begin{cases}
    1 & i=j\\
    0 & \text{else}
    \end{cases}$. 
\end{problem}
\begin{solution}{}
    Note that 
    \begin{align*}
        \frac{\partial y_{r}}{\partial W_{ij}} = \delta_{rj}\frac{\partial y_{r}}{\partial W_{ij}}
    \end{align*}
    And
    \begin{align*}
        \frac{\partial y_{j}}{\partial W_{ij}} = x_j
    \end{align*}
    Then
    \begin{align*}
        \frac{\partial J}{\partial W_{ij}}&=\sum_{r=1}^{m}\frac{\partial J}{\partial y_{r}}\frac{\partial y_{r}}{\partial W_{ij}}.\\
        &= \frac{\partial J}{\partial y_{j}}\frac{\partial y_{j}}{\partial W_{ij}}\\
        &= \frac{\partial J}{\partial y_{j}}x_j
    \end{align*}
\end{solution}
\newpage

\begin{problem}{4.1.1.2}
    Now let's vectorize this. Let's write $\frac{\partial J}{\partial y}\in\reals^{m\times1}$
    for the column vector whose $i$th entry is $\frac{\partial J}{\partial y_{i}}$.
    Let's also define the matrix $\frac{\partial J}{\partial W}\in\reals^{m\times d}$,
    whose $ij$'th entry is $\frac{\partial J}{\partial W_{ij}}$. Generally
    speaking, we'll always take $\frac{\partial J}{\partial A}$ to be
    an array of the same size (``shape'' in numpy) as $A$. Give a vectorized
    expression for $\frac{\partial J}{\partial W}$ in terms of the column
    vectors $\frac{\partial J}{\partial y}$ and $x$. {[}Hint: Outer
    product.{]} 
\end{problem}
\begin{solution}{}
    \begin{align*}
        \frac{\partial J}{\partial W} = \frac{\partial J}{\partial y} \bigotimes x
    \end{align*}
\end{solution}
\newpage

\begin{problem}{4.1.1.3}
    In the usual way, define $\frac{\partial J}{\partial x}\in\reals^{d}$,
    whose $i$'th entry is $\frac{\partial J}{\partial x_{i}}$. Show
    that 
    \[
    \frac{\partial J}{\partial x}=W^{T}\left(\frac{\partial J}{\partial y}\right)
    \]
\end{problem}
\begin{solution}
    Let $W_i$ denotes the ith column of $W$.
    
    Then observe that
    \begin{align*}
        \frac{\partial J}{\partial x_i} &= \sum_{r=1}^m \frac{\partial J}{\partial y_r}\frac{\partial y_r}{\partial x_i}\\
        &=\sum_{r=1}^m \frac{\partial J}{\partial y_r}W_{r,i}\\
        &= \left(W_i\right)^T\frac{\partial J}{\partial y} 
    \end{align*}
    Thus
    \begin{align*}
        \frac{\partial J}{\partial x} = W^T\frac{\partial J}{\partial y}
    \end{align*}
\end{solution}
\newpage

\begin{problem}{4.1.1.4}
    Show that $\frac{\partial J}{\partial b}=\frac{\partial J}{\partial y}$,
    where $\frac{\partial J}{\partial b}$ is defined in the usual way.
\end{problem}
\begin{solution}{}
    Note that $\frac{\partial y}{\partial b} = 1$. Then
    \begin{align*}
        \frac{\partial J}{\partial b} = \frac{\partial J}{\partial y}\frac{\partial y}{\partial b} = \frac{\partial J}{\partial y}
    \end{align*}
\end{solution}
\newpage

\begin{problem}{4.1.2}
    Show that $\frac{\partial J}{\partial A}=\frac{\partial J}{\partial S}\odot\sigma'(A)$,
    where we're using $\odot$ to represent the \textbf{Hadamard product}.
    If $A$ and $B$ are arrays of the same shape, then their Hadamard
    product $A\odot B$ is an array with the same shape as $A$ and $B$,
    and for which $\left(A\odot B\right)_{i}=A_{i}B_{i}$. That is, it's
    just the array formed by multiplying corresponding elements of $A$
    and $B$. Conveniently, in \code{numpy} if \code{A} and \code{B}
    are arrays of the same shape, then \code{A{*}B} is their Hadamard
    product.
\end{problem}
\begin{solution}{}
    Because of how $\sigma ()$ is defined, $\frac{\partial S_i}{\partial A_i} = \sigma '(A_i)$.
    
    Thus $\frac{\partial J}{\partial A_i} = \frac{\partial J}{\partial S_i}\frac{\partial S_i}{\partial A_i} = \frac{\partial J}{\partial S_i}\sigma '(A_i)$.
    
    So then $\frac{\partial J}{\partial A}=\frac{\partial J}{\partial S}\odot \sigma '(A)$.
\end{solution}
\end{document}
